\documentclass[12pt,a4paper]{article}
\usepackage[utf8]{inputenc}
%\usepackage[slovak]{babel}
\usepackage[IL2]{fontenc}
\usepackage{amsmath}
\usepackage{amsfonts}
\usepackage{amssymb}

%\title{Automatizovaná detekce a klasifikace jevů v časových řadách hydraulických sensorů}
\title{Automatized detection and classification of events in the time series of hydraulic sensors}
\author{Alexander Ma\v{c}ejovsk\'{y} \and \v{S}t\v{e}p\'{a}n Pardubick\'{y}}
\date{\today}

\begin{document}

%%\begin{titlepage}
%%    \centering
%%    \vspace*{2cm}
%%    {\huge\bfseries \maketitle}
%%    \vfill
   % \emph{Last edited on:} \today % Date will update automatically
%%    \vfill
%%\end{titlepage}

\maketitle % Insert title here

\thispagestyle{empty} % Remove page number from title page

\clearpage % Start the table of contents on a new page

\tableofcontents  

\newpage



\section{Introduction}
we have time  series of flow, level and velocity coming from water canals and pipes

assignment - detection and correction, 

condition - only univariate time series itself can be considered

there is no target, no example of what errors look like nor of what corrected time series looks like

output - python library for automatized performing of the assigned tasks

The domain of sewage management is critical for urban infrastructure, but it is also often overlooked.
The traditional methods of monitoring sewage systems are constrained by the limitations of manual analysis, often resulting in delayed responses to anomalies and missed opportunities for optimization.
Our goal was to create a python library that would help in the automation of this process.



\section{Data}




\section{Identified Events}




\section{Corrections of Errors}


\section{Performance Evaluation}
no target $\Rightarrow$ not clear how to do this, but detection of events done expertly with the help of consultations,  basic idea for evaluation of corrections could be that no (only few) errors should be detected in already corrected time series 


\section{Python Library}

very brief summary of what the resulting library contains, how it is used, + link to the github page (we should probably restrict it with a password)






\section{Conclusion}





\end{document}